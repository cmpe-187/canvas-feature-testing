\documentclass[10pt,letterpaper]{article}
\usepackage[utf8]{inputenc}
\usepackage{amsmath}
\usepackage{amsfonts}
\usepackage{amssymb}

\usepackage[margin=0.5in]{geometry}
\usepackage{graphicx}
\usepackage{tabularx}
\usepackage{booktabs}
\pagenumbering{gobble}

%compact lists
\usepackage{enumitem}
\setitemize{noitemsep,topsep=0pt,parsep=0pt,partopsep=0pt}

\title{Feature Testing}
\author{
	Cai, Zelin\\
	\and
	Silvestre, Patrick\\
}
\date{}

\begin{document}
\maketitle
\section{Calendar - Add Event as Student (Simplified Version)}
\subsection{Input Domain}
The input domain of this features includes the title of the event, a date, a starting time, a ending time, a location, and a calendar. The input domain can be partitioned: e.g. one partition for title, one partition for date, etc.

\begin{figure}[h]
	\centerline{\includegraphics[width=10cm]{screenshots/edit-event.png}}
	\caption{Add Event Interface}
\end{figure}

\newpage
\subsubsection{Title}
The input domain of the input title is a string, which may be empty.

\subsubsection{Date}
The input domain of the date is a string in the format \texttt{YYYY-MM-DD}, where \texttt{YYYY} represents a year, \texttt{MM} represents a month, and \texttt{DD} represents a day. The month and day fields need not be padded with zeros. Alternatively, a user can opt to input a date using a GUI in the form of a calendar.

\subsubsection{Starting Time}
The input domain of the starting time is a string in the format \texttt{HH:MMxx}, where \texttt{HH} represents hours, \texttt{MM} represents minutes, and \texttt{xx} represents AM or PM (case insensitive). The hour and minute fields need not be padded with zeros.

\subsubsection{Ending Time}
The input domain of the ending time is a string in the format \texttt{HH:MMxx}, where \texttt{HH} represents hours, \texttt{MM} represents minutes, and \texttt{xx} represents AM or PM (case insensitive). The hour and minute fields need not be padded with zeros.

\subsubsection{Location}
The input domain of the input title is a string, which may be empty.

\subsubsection{Calendar}
The input domain of the calendar is a list of the user's calendars, of which one can be selected.

\newpage
\subsection{Test Cases}
\begin{table}[h!]
\begin{tabularx}{\textwidth}{lXXXl}
\toprule
TC \# &
  Input &
  Expected Output &
  Actual Output &
  Tess Pass/Fail \\ \midrule
1 &
  \begin{itemize}
    \item{title: (empty)}
    \item{date: 2020-5-2}
    \item{starting time: 12:00am}
    \item{ending time: 12:00am}
    \item{location: (empty)}
    \item{calendar: Patrick Silvestre}
    \item{UI: press submit button}
  \end{itemize} &
  Patrick Silvestre's calendar displays Untitled event on 2020-5-2. Event has no starting nor ending time (all-day event) &
  See figure 2. &
  Pass \\ \midrule
2 &
  \begin{itemize}
    \item{title: Test}
    \item{date: 2020-5-2}
    \item{starting time: 12:00am}
    \item{ending time: 12:00am}
    \item{location: Test}
    \item{calendar: Patrick Silvestre}
    \item{UI: press submit button}
  \end{itemize} &
  Patrick Silvestre's calendar displays Test event on 2020-5-2 with Test location. Event has no starting nor ending time (all-day event). &
  See figure 3. &
  Pass \\ \midrule
3 &
  \begin{itemize}
    \item{title: (empty)}
    \item{date: 2020-5-2}
    \item{starting time: 12:00pm}
    \item{ending time: 1:00pm}
    \item{location: (empty)}
    \item{calendar: Patrick Silvestre}
    \item{UI: press submit button}
  \end{itemize} &
  Patrick Silvestre's calendar displays Test event on 2020-5-2 with test location. Event has starting time 12:00pm and ending time 1:00pm). &
  See figure 4. &
  Pass \\ \midrule
4 &
  \begin{itemize}
    \item{title: Test}
    \item{date: 2020-4-25}
    \item{starting time: 12:00am}
    \item{ending time: 12:00am}
    \item{location: Test}
    \item{calendar: Patrick Silvestre}
    \item{UI: press submit button}
  \end{itemize} &
  Patrick Silvestre's calendar displays Test event on 2020-4-25 with test location. Event has no starting nor ending time (all-day event). &
  See figure 5. &
  Pass \\ \midrule
5 &
  \begin{itemize}
    \item{title: Test}
    \item{date: 2020-4-25}
    \item{starting time: 12:00pm}
    \item{ending time: 11:00am}
    \item{location: Test}
    \item{calendar: Patrick Silvestre}
    \item{UI: press submit button}
  \end{itemize} &
  Add Event interface warns user that event ends before it starts and prevents user from submitting event. &
  System automatically adjusts either starting or ending time to prevent invalid times with no warning, implicitly allowing user to submit event. See figure 6. &
  Fail \\ \bottomrule
\end{tabularx}
\end{table}

\newpage
\begin{figure}[h!]
	\centerline{\includegraphics[width=8cm]{screenshots/tc01-actual-output.png}}
	\caption{TC-01 actual output}
\end{figure}
\begin{figure}[h!]
	\centerline{\includegraphics[width=8cm]{screenshots/tc02-actual-output.png}}
	\caption{TC-02 actual output}
\end{figure}
\begin{figure}[h!]
	\centerline{\includegraphics[width=8cm]{screenshots/tc03-actual-output.png}}
	\caption{TC-03 actual output}
\end{figure}
\begin{figure}[h!]
	\centerline{\includegraphics[width=8cm]{screenshots/tc04-actual-output.png}}
	\caption{TC-04 actual output}
\end{figure}
\begin{figure}[h!]
	\centerline{\includegraphics[width=8cm]{screenshots/tc04-actual-output.png}}
	\caption{TC-04 actual output}
\end{figure}

\newpage
\section{Inbox - Composing a Message}
\subsection{Input Domain}
The input domain for the composing a message includes the Course section, the message recipients (only available after selecting the Course section), the subject of the message, the body of the message, any attachments the user wants to add to the message, and the option to send a different message to each recipient (This option will not be covered in the test cases).
Most of the input domain can be partitioned (One for the subject, one for the attachments, etc.) except for selecting the message recipient because the user must select a course before they can select their recipients.

\begin{figure}[h!]
	\centerline{\includegraphics[width=12cm]{screenshots/compose-message-before-selecting-course.png}}
	\caption{Before selecting a course}
\end{figure}
\begin{figure}[h!]
	\centerline{\includegraphics[width=12cm]{screenshots/compose-message-after-selecting-course.png}}
	\caption{After selecting a course}
\end{figure}
\begin{figure}[h!]
	\centerline{\includegraphics[width=8cm]{screenshots/compose-message-attachment.png}}
	\caption{Attachment to be used in all applicable tests}
\end{figure}

\subsubsection{Course}
The input domain is a selection from a list of courses that the user is currently enrolled in.

\subsubsection{To}
The input domain is a selection from a list of users that are in the selected course.

\subsubsection{Subject}
The input domain is a string input.

\subsubsection{Message Body}
The input domain is a string input.

\subsubsection{Attachments}
The input domain is a file either from the user’s files on Canvas or on their computer.

\newpage
\subsection{Test Cases}
\begin{table}[h!]
\begin{tabularx}{\textwidth}{lXXXl}
\toprule
TC \# &
  Input &
  Expected Output &
  Actual Output &
  Tess Pass/Fail \\ \midrule
6 &
  \begin{itemize}
    \item{course: CMPE-187}
    \item{to: (empty)}
    \item{subject: Test1}
    \item{message: Hello there}
    \item{attachments: (empty)}
  \end{itemize} &
   Canvas informs user that recipient is invalid. &
   See figure 10. &
   Pass \\ \midrule
7 &
  \begin{itemize}
    \item{course: CMPE-187}
    \item{to: Zelin Cai}
    \item{subject: Test2}
    \item{message: (empty)}
    \item{attachments: (empty)}
  \end{itemize} &
   Canvas sends the message normally &
   See figure 11. &
   Fail \\ \midrule
8 &
  \begin{itemize}
    \item{course: CMPE-187}
    \item{to: Zelin Cai}
    \item{subject: (empty)}
    \item{message: Test Case 3}
    \item{attachments: (empty)}
  \end{itemize} &
   Canvas sends the message normally &
   Canvas outputs message title "(No Title)". See figure 12. &
   Pass \\ \midrule
9 &
  \begin{itemize}
    \item{course: CMPE-187}
    \item{to: Zelin Cai}
    \item{subject: Test4}
    \item{message: Testing Case 4}
    \item{attachments: A \texttt{.rar} file, approximately 80MB}
  \end{itemize} &
   Canvas informs the user that the file is too large for attachment. &
   Canvas sends message after about 10 minutes. See figure 13. &
   Fail \\ \midrule
10 &
  \begin{itemize}
    \item{course: (empty)}
    \item{to: (empty)}
    \item{subject: Test5}
    \item{message: Testing Case 5}
    \item{attachments: (empty)}
  \end{itemize} &
   Canvas informs user that the course selection is invalid. &
   Canvas output is identical to TC-06. See figure 14. &
   Fail \\ \bottomrule
\end{tabularx}
\end{table}

\begin{figure}[h!]
	\centerline{\includegraphics[width=12cm]{screenshots/tc06-actual-output.png}}
	\caption{TC-06 actual output}
\end{figure}
\begin{figure}[h!]
	\centerline{\includegraphics[width=12cm]{screenshots/tc07-actual-output.png}}
	\caption{TC-07 actual output}
\end{figure}
\begin{figure}[h!]
	\centerline{\includegraphics[width=12cm]{screenshots/tc08-actual-output.png}}
	\caption{TC-08 actual output}
\end{figure}
\begin{figure}[h!]
	\centerline{\includegraphics[width=12cm]{screenshots/tc09-actual-output.png}}
	\caption{TC-09 actual output}
\end{figure}
\begin{figure}[h!]
	\centerline{\includegraphics[width=12cm]{screenshots/tc10-actual-output.png}}
	\caption{TC-10 actual output}
\end{figure}

\end{document}
